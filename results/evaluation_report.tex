\documentclass[11pt,a4paper]{article}
\usepackage[utf8]{inputenc}
\usepackage[T1]{fontenc}
\usepackage{geometry}
\geometry{margin=1in}
\usepackage{booktabs}
\usepackage{longtable}
\usepackage{enumitem}
\usepackage{xcolor}
\usepackage{hyperref}
\usepackage{fancyhdr}
\usepackage{titlesec}
\usepackage{amssymb}

\definecolor{correct}{RGB}{34,139,34}
\definecolor{wrong}{RGB}{200,30,30}
\definecolor{neutral}{RGB}{80,80,80}

\pagestyle{fancy}
\fancyhf{}
\fancyhead[L]{Plant Disease Classification Report}
\fancyhead[R]{\thepage}
\renewcommand{\headrulewidth}{0.4pt}

\title{Plant Disease Classification:\\Baseline vs Agent Evaluation Report}
\author{Generated by \texttt{generate\_report.py}}

\date{2026-02-12 09:31:11}

\begin{document}
\maketitle
\tableofcontents
\newpage


\section{Executive Summary}

This report compares two approaches for plant disease image classification using Claude:
\begin{itemize}
  \item \textbf{Baseline}: Direct API call to Claude Haiku with only the image and class list (single inference, no context).
  \item \textbf{Agent}: Claude Code headless mode (\texttt{claude -p}) where the agent reads a symptom knowledge base, views the test image, views reference images, reasons through candidates, and then predicts.
\end{itemize}


\subsection{Overall Results}
\begin{table}[h]
\centering
\begin{tabular}{lccc}
\toprule
\textbf{Dataset} & \textbf{Baseline} & \textbf{Agent} & \textbf{Improvement} \\
\midrule

Foliar\_Disease\_Stress & 42.9\% & 64.3\% & \textcolor{correct}{+21.4\%} \\

\bottomrule
\end{tabular}
\caption{Overall accuracy comparison across datasets.}
\end{table}


\subsection{Agent Cost and Latency}
\begin{table}[h]
\centering
\begin{tabular}{lrrrrr}
\toprule
\textbf{Dataset} & \textbf{Images} & \textbf{Avg Turns} & \textbf{Avg Duration (s)} & \textbf{Avg Cost (\$)} & \textbf{Total Cost (\$)} \\
\midrule

Foliar\_Disease\_Stress & 14 & 7.6 & 12.4 & 0.0220 & 0.3077 \\

\bottomrule
\end{tabular}
\caption{Agent execution statistics.}
\end{table}
\newpage

\section{Foliar\_Disease\_Stress: Mango Leaf Diseases}

\subsection{Per-Class Accuracy}
\begin{table}[h]
\centering
\begin{tabular}{lcccc}
\toprule
\textbf{Class} & \textbf{Baseline} & \textbf{Agent} & \textbf{Diff} \\
\midrule

Anthracnose & 2/2 & 2/2 & 0\% \\
Bacterial\_Canker & 0/2 & 1/2 & \textcolor{correct}{+50\%} \\
Cutting\_Weevil & 1/2 & 2/2 & \textcolor{correct}{+50\%} \\
Die\_Back & 1/2 & 1/2 & 0\% \\
Gall\_Midge & 0/2 & 1/2 & \textcolor{correct}{+50\%} \\
Powdery\_Mildew & 0/2 & 0/2 & 0\% \\
Sooty\_Mould & 2/2 & 2/2 & 0\% \\

\bottomrule
\end{tabular}
\caption{Per-class accuracy for Foliar\_Disease\_Stress.}
\end{table}


\subsection{Image-by-Image Results}
\begin{longtable}{llllcllc}
\toprule
\textbf{\#} & \textbf{Image} & \textbf{Ground Truth} & \textbf{Baseline} & & \textbf{Agent} & & \textbf{Turns} \\
\midrule
\endhead

1 & \texttt{test\_001.jpg} & Die\_Back & Gall\_Midge & \textcolor{wrong}{$\times$} & Die\_Back & \textcolor{correct}{$\checkmark$} & 9 \\
2 & \texttt{test\_002.jpg} & Cutting\_Weevil & Die\_Back & \textcolor{wrong}{$\times$} & Cutting\_Weevil & \textcolor{correct}{$\checkmark$} & 6 \\
3 & \texttt{test\_003.jpg} & Powdery\_Mildew & Anthracnose & \textcolor{wrong}{$\times$} & Sooty\_Mould & \textcolor{wrong}{$\times$} & 7 \\
4 & \texttt{test\_004.jpg} & Bacterial\_Canker & Anthracnose & \textcolor{wrong}{$\times$} & Bacterial\_Canker & \textcolor{correct}{$\checkmark$} & 9 \\
5 & \texttt{test\_005.jpg} & Cutting\_Weevil & Cutting\_Weevil & \textcolor{correct}{$\checkmark$} & Cutting\_Weevil & \textcolor{correct}{$\checkmark$} & 8 \\
6 & \texttt{test\_006.jpg} & Powdery\_Mildew & Sooty\_Mould & \textcolor{wrong}{$\times$} & Sooty\_Mould & \textcolor{wrong}{$\times$} & 6 \\
7 & \texttt{test\_007.jpg} & Anthracnose & Anthracnose & \textcolor{correct}{$\checkmark$} & Anthracnose & \textcolor{correct}{$\checkmark$} & 8 \\
8 & \texttt{test\_008.jpg} & Die\_Back & Die\_Back & \textcolor{correct}{$\checkmark$} & Cutting\_Weevil & \textcolor{wrong}{$\times$} & 6 \\
9 & \texttt{test\_009.jpg} & Sooty\_Mould & Sooty\_Mould & \textcolor{correct}{$\checkmark$} & Sooty\_Mould & \textcolor{correct}{$\checkmark$} & 8 \\
10 & \texttt{test\_010.jpg} & Sooty\_Mould & Sooty\_Mould & \textcolor{correct}{$\checkmark$} & Sooty\_Mould & \textcolor{correct}{$\checkmark$} & 9 \\
11 & \texttt{test\_011.jpg} & Anthracnose & Anthracnose & \textcolor{correct}{$\checkmark$} & Anthracnose & \textcolor{correct}{$\checkmark$} & 8 \\
12 & \texttt{test\_012.jpg} & Gall\_Midge & Anthracnose & \textcolor{wrong}{$\times$} & Gall\_Midge & \textcolor{correct}{$\checkmark$} & 7 \\
13 & \texttt{test\_013.jpg} & Gall\_Midge & Anthracnose & \textcolor{wrong}{$\times$} & Sooty\_Mould & \textcolor{wrong}{$\times$} & 8 \\
14 & \texttt{test\_014.jpg} & Bacterial\_Canker & Anthracnose & \textcolor{wrong}{$\times$} & Sooty\_Mould & \textcolor{wrong}{$\times$} & 8 \\

\bottomrule
\end{longtable}

\subsection{Agent Reasoning Traces}

\subsubsection{Correct Predictions}

\paragraph{test\_001.jpg — \textcolor{correct}{Correct}}
Ground Truth: \textbf{Die\_Back} $\rightarrow$ Prediction: \textbf{Die\_Back} \\
Turns: 9 \quad Duration: 12.9s \quad Cost: \$0.0298
\begin{description}[style=unboxed,leftmargin=0.5cm]
\item[\textbf{Think:}] I'll help you classify the plant disease in the image. Let me start by reading the knowledge base and examining the target image.
\item[\textbf{Tool:}] \texttt{Read} $\rightarrow$ \texttt{disease\_symptoms.md}
\item[\textbf{Tool:}] \texttt{Read} $\rightarrow$ \texttt{test\_data/Foliar\_Disease\_Stress/images/test\_001.jpg}
\item[\textbf{Think:}] Good! I can see the target image shows a mango leaf with dark discoloration. Now let me examine the reference images for the most likely candidates. Based on the visual features I see (dark brown/black appearance on the leaf), the top candidates appear to be:

1. **Anthracnose** - dark flat sunken lesions
2. **Sooty Mould** - dark coating appearance
3. **Die Back** - browning and wilting

Let me view the reference images for these candidates:
\item[\textbf{Tool:}] \texttt{Read} $\rightarrow$ \texttt{Plant\_Disease\_Dataset/Foliar\_Disease\_Stress/Anthracnose/Anthracnose\_001.jpg}
\item[\textbf{Tool:}] \texttt{Read} $\rightarrow$ \texttt{Plant\_Disease\_Dataset/Foliar\_Disease\_Stress/Sooty\_Mould/Sooty\_Mould\_001.jpg}
\item[\textbf{Tool:}] \texttt{Read} $\rightarrow$ \texttt{Plant\_Disease\_Dataset/Foliar\_Disease\_Stress/Die\_Back/Die\_Back\_001.jpg}
\item[\textbf{Think:}] Let me view more reference images to better compare:
\item[\textbf{Tool:}] \texttt{Read} $\rightarrow$ \texttt{Plant\_Disease\_Dataset/Foliar\_Disease\_Stress/Anthracnose/Anthracnose\_002.jpg}
\item[\textbf{Tool:}] \texttt{Read} $\rightarrow$ \texttt{Plant\_Disease\_Dataset/Foliar\_Disease\_Stress/Sooty\_Mould/Sooty\_Mould\_002.jpg}
\item[\textbf{Tool:}] \texttt{Read} $\rightarrow$ \texttt{Plant\_Disease\_Dataset/Foliar\_Disease\_Stress/Die\_Back/Die\_Back\_002.jpg}
\item[\textbf{Think:}] Based on my analysis of the target image and comparison with the reference images:

**Visual Analysis:**

The target image (test\_001.jpg) shows a mango leaf that is predominantly dark brown/black in color. Comparing this to the reference images:

- **Anthracnose references** (Anthracnose\_001, \_002): Show leaves with scattered dark lesions and brown patches on a mostly tan/brown background, with some holes and eaten-away margins on otherwise visible leaf structure.

- **Sooty Mould references** (Sooty\_Mould\_001, \_002): Show green leaves with dark speckles/coating distributed across the surface - the leaf underneath remains visibly green.

- **Die Back references** (Die\_Back\_001, \_002): Show completely darkened/withered leaves that are uniformly brown/black, wilted, and desiccated.

The target image matches the **Die Back** pattern most closely - it shows:
- Uniform dark brown/black discoloration of the entire leaf
- A wilted, withered appearance
- Loss of normal leaf structure and vitality
- The leaf appears completely desiccated, not showing the scattered lesion patterns of Anthracnose or the visible green background with dark specks of Sooty Mould

```json
\{"prediction": "Die\_Back"\}
```
\end{description}
\medskip\hrule\medskip
\paragraph{test\_002.jpg — \textcolor{correct}{Correct}}
Ground Truth: \textbf{Cutting\_Weevil} $\rightarrow$ Prediction: \textbf{Cutting\_Weevil} \\
Turns: 6 \quad Duration: 9.1s \quad Cost: \$0.0183
\begin{description}[style=unboxed,leftmargin=0.5cm]
\item[\textbf{Think:}] I'll help you classify the plant disease in the image. Let me start by reading the knowledge base and examining the target image.
\item[\textbf{Tool:}] \texttt{Read} $\rightarrow$ \texttt{disease\_symptoms.md}
\item[\textbf{Tool:}] \texttt{Read} $\rightarrow$ \texttt{test\_data/Foliar\_Disease\_Stress/images/test\_002.jpg}
\item[\textbf{Think:}] Good! I can see the target image shows leaves that appear to be curled, twisted, and contorted with a distinctive mechanical damage pattern. The leaves look like they have been cut or scraped. This is characteristic of Cutting Weevil damage. However, let me examine the reference images for my top candidate classes to make a thorough comparison.
\item[\textbf{Tool:}] \texttt{Read} $\rightarrow$ \texttt{Plant\_Disease\_Dataset/Foliar\_Disease\_Stress/Cutting\_Weevil/Cutting\_Weevil\_001.jpg}
\item[\textbf{Tool:}] \texttt{Read} $\rightarrow$ \texttt{Plant\_Disease\_Dataset/Foliar\_Disease\_Stress/Cutting\_Weevil/Cutting\_Weevil\_002.jpg}
\item[\textbf{Tool:}] \texttt{Read} $\rightarrow$ \texttt{Plant\_Disease\_Dataset/Foliar\_Disease\_Stress/Die\_Back/Die\_Back\_001.jpg}
\item[\textbf{Think:}] Perfect! Comparing the target image with the reference images:

**Analysis:**

1. **Target Image (test\_002.jpg)**: Shows leaves that are curled, twisted, contorted, and appear to have been cut or stripped. The leaves hang from the twigs in a characteristic manner.

2. **Cutting Weevil Reference Images**: 
   - Cutting\_Weevil\_001.jpg shows brown, curled, and contorted leaves typical of weevil damage
   - Cutting\_Weevil\_002.jpg shows stripped, leafless shoots with few remaining leaves

3. **Die\_Back Reference Image**: Shows completely brown/dead leaves that are curled but in a wilted, dried state from disease progression

**Key Observations:**
- The target image shows mechanical damage patterns (cuts, stripped areas) rather than disease lesions
- Leaves are curled and twisted, matching Cutting Weevil damage patterns
- The damage is from scraping and cutting by adult weevils, not from fungal or bacterial disease
- This matches the description: "brown, curled, and crumpled young leaves caused by adult weevils scraping the surface layers"

```json
\{"prediction": "Cutting\_Weevil"\}
```
\end{description}
\medskip\hrule\medskip
\paragraph{test\_004.jpg — \textcolor{correct}{Correct}}
Ground Truth: \textbf{Bacterial\_Canker} $\rightarrow$ Prediction: \textbf{Bacterial\_Canker} \\
Turns: 9 \quad Duration: 12.8s \quad Cost: \$0.0220
\begin{description}[style=unboxed,leftmargin=0.5cm]
\item[\textbf{Think:}] I'll help you classify the disease in the mango leaf image. Let me start by reading the knowledge base and examining the target image.
\item[\textbf{Tool:}] \texttt{Read} $\rightarrow$ \texttt{disease\_symptoms.md}
\item[\textbf{Tool:}] \texttt{Read} $\rightarrow$ \texttt{test\_data/Foliar\_Disease\_Stress/images/test\_004.jpg}
\item[\textbf{Think:}] Good! I can see the target image shows a green mango leaf with what appears to be dark spots or damage on the edges. The leaf has a relatively green background but with visible brown/dark damage along the margins.

Based on the initial visual inspection, my top candidates are:
1. **Anthracnose** - shows dark lesions with eaten-away margins
2. **Bacterial Canker** - angular lesions with yellow halos
3. **Sooty Mould** - dark coating on leaf surface

Let me examine the reference images for these top candidates to make a more accurate comparison.
\item[\textbf{Tool:}] \texttt{Read} $\rightarrow$ \texttt{Plant\_Disease\_Dataset/Foliar\_Disease\_Stress/Anthracnose/Anthracnose\_001.jpg}
\item[\textbf{Tool:}] \texttt{Read} $\rightarrow$ \texttt{Plant\_Disease\_Dataset/Foliar\_Disease\_Stress/Anthracnose/Anthracnose\_002.jpg}
\item[\textbf{Tool:}] \texttt{Read} $\rightarrow$ \texttt{Plant\_Disease\_Dataset/Foliar\_Disease\_Stress/Bacterial\_Canker/Bacterial\_Canker\_001.jpg}
\item[\textbf{Tool:}] \texttt{Read} $\rightarrow$ \texttt{Plant\_Disease\_Dataset/Foliar\_Disease\_Stress/Bacterial\_Canker/Bacterial\_Canker\_002.jpg}
\item[\textbf{Tool:}] \texttt{Read} $\rightarrow$ \texttt{Plant\_Disease\_Dataset/Foliar\_Disease\_Stress/Sooty\_Mould/Sooty\_Mould\_001.jpg}
\item[\textbf{Tool:}] \texttt{Read} $\rightarrow$ \texttt{Plant\_Disease\_Dataset/Foliar\_Disease\_Stress/Sooty\_Mould/Sooty\_Mould\_002.jpg}
\item[\textbf{Think:}] Now I can clearly compare the images. Let me analyze:

**Target Image (test\_004.jpg):** 
- Green leaf with dark brown/black damage along the edges
- Appears to have eaten-away or irregular margins
- The damage looks sunken and embedded in the tissue
- Dark spots with relatively clean margins along the leaf edge

**Comparison Analysis:**

1. **Anthracnose References:**
   - Anthracnose\_001.jpg: Shows a tan/brown leaf with darker sunken lesions - very different from target (leaf is more brown overall)
   - Anthracnose\_002.jpg: Shows a green leaf with small dark spots/lesions scattered across it - similar pattern to target
   - Bacterial\_Canker\_001.jpg: Shows a green leaf with dark patches along the margins with irregular edges - **VERY SIMILAR to test\_004**
   - Bacterial\_Canker\_002.jpg: Identical to test\_004 - this is a very close match

2. **Bacterial Canker References:**
   - These show angular lesions with affected leaf margins, matching the target well

3. **Sooty Mould References:**
   - Sooty\_Mould\_001.jpg: Shows dark coating on lower leaf surface - darker overall
   - Sooty\_Mould\_002.jpg: Shows scattered small speckles across the leaf surface - different pattern from target

**Key Observation:** 
The target image (test\_004.jpg) is **visually identical to Bacterial\_Canker\_002.jpg**. Both show:
- Green leaf background
- Dark brown/black damage concentrated along the leaf margins
- Irregular, eaten-away edge appearance
- The damage appears to follow the leaf edge contours

The disease in test\_004 matches the Bacterial Canker symptom description: "raised, dark appearance" with "damage along the margins" and the distinctive pattern of lesions constraining to affect the leaf edges.

```json
\{"prediction": "Bacterial\_Canker"\}
```
\end{description}
\medskip\hrule\medskip
\paragraph{test\_005.jpg — \textcolor{correct}{Correct}}
Ground Truth: \textbf{Cutting\_Weevil} $\rightarrow$ Prediction: \textbf{Cutting\_Weevil} \\
Turns: 8 \quad Duration: 11.9s \quad Cost: \$0.0229
\begin{description}[style=unboxed,leftmargin=0.5cm]
\item[\textbf{Think:}] I'll help you classify the mango leaf disease. Let me start by reading the knowledge base and examining the images.
\item[\textbf{Tool:}] \texttt{Read} $\rightarrow$ \texttt{disease\_symptoms.md}
\item[\textbf{Tool:}] \texttt{Read} $\rightarrow$ \texttt{test\_data/Foliar\_Disease\_Stress/images/test\_005.jpg}
\item[\textbf{Think:}] Now let me view reference images for the diseases that could match this image. The image shows what appears to be leaves with cutting/mechanical damage. Let me check the Cutting Weevil reference images first, as well as a couple other candidates:
\item[\textbf{Tool:}] \texttt{Read} $\rightarrow$ \texttt{Plant\_Disease\_Dataset/Foliar\_Disease\_Stress/Cutting\_Weevil/Cutting\_Weevil\_001.jpg}
\item[\textbf{Tool:}] \texttt{Read} $\rightarrow$ \texttt{Plant\_Disease\_Dataset/Foliar\_Disease\_Stress/Cutting\_Weevil/Cutting\_Weevil\_002.jpg}
\item[\textbf{Tool:}] \texttt{Read} $\rightarrow$ \texttt{Plant\_Disease\_Dataset/Foliar\_Disease\_Stress/Cutting\_Weevil/Cutting\_Weevil\_003.jpg}
\item[\textbf{Think:}] Perfect! The test\_005.jpg image is nearly identical to Cutting\_Weevil\_003.jpg. Both show:
- Young leaves that are green and intact
- Leaves that appear partially severed or cut near the base
- A contorted/crumpled appearance
- Stripped, leafless shoot patterns
- The characteristic cutting pattern from adult weevils

Let me also check Die\_Back to be thorough, as it can sometimes involve defoliation:
\item[\textbf{Tool:}] \texttt{Read} $\rightarrow$ \texttt{Plant\_Disease\_Dataset/Foliar\_Disease\_Stress/Die\_Back/Die\_Back\_001.jpg}
\item[\textbf{Tool:}] \texttt{Read} $\rightarrow$ \texttt{Plant\_Disease\_Dataset/Foliar\_Disease\_Stress/Die\_Back/Die\_Back\_002.jpg}
\item[\textbf{Think:}] \#\# Analysis

**Test Image (test\_005.jpg):** Shows young, green mango leaves that are partially severed or cut near the base, with contorted leaf shapes and a stripped, bare shoot pattern.

**Key Observations:**
- The leaves are intact and green (not brown or diseased-looking)
- Multiple leaves appear cut or partially detached near their base/petiole
- The shoot shows a pattern of stripped, bare sections
- This matches the mechanical injury pattern described for Cutting Weevil

**Comparison with Reference Images:**
- **Cutting\_Weevil\_003.jpg:** Nearly identical to the test image - same green leaves, same cutting patterns, same contorted appearance
- **Die\_Back Images:** Show brown/necrotic tissue, withered leaves, and bark discoloration - NOT present in test image

**Diagnosis:** The image clearly shows **Cutting Weevil** damage, characterized by:
- Leaves cut near the base by female weevils
- Young, green leaves with mechanical injury (not disease discoloration)
- Contorted, crumpled leaf shapes
- Stripped shoot pattern

```json
\{"prediction": "Cutting\_Weevil"\}
```
\end{description}
\medskip\hrule\medskip
\paragraph{test\_007.jpg — \textcolor{correct}{Correct}}
Ground Truth: \textbf{Anthracnose} $\rightarrow$ Prediction: \textbf{Anthracnose} \\
Turns: 8 \quad Duration: 14.0s \quad Cost: \$0.0237
\begin{description}[style=unboxed,leftmargin=0.5cm]
\item[\textbf{Think:}] I'll help you classify the disease in the mango leaf image. Let me start by reading the knowledge base and viewing the target image.
\item[\textbf{Tool:}] \texttt{Read} $\rightarrow$ \texttt{disease\_symptoms.md}
\item[\textbf{Tool:}] \texttt{Read} $\rightarrow$ \texttt{test\_data/Foliar\_Disease\_Stress/images/test\_007.jpg}
\item[\textbf{Think:}] Now let me view reference images for the most likely candidates. Based on the target image showing a leaf with scattered reddish-brown spots/patches on a tan-colored leaf, my top candidates are:
1. Anthracnose - for the sunken, embedded lesions with darker edges
2. Sooty Mould - for the dusty/speckled appearance
3. Gall Midge - for the raised bumps

Let me examine reference images:
\item[\textbf{Tool:}] \texttt{Read} $\rightarrow$ \texttt{Plant\_Disease\_Dataset/Foliar\_Disease\_Stress/Anthracnose/Anthracnose\_001.jpg}
\item[\textbf{Tool:}] \texttt{Read} $\rightarrow$ \texttt{Plant\_Disease\_Dataset/Foliar\_Disease\_Stress/Anthracnose/Anthracnose\_002.jpg}
\item[\textbf{Tool:}] \texttt{Read} $\rightarrow$ \texttt{Plant\_Disease\_Dataset/Foliar\_Disease\_Stress/Gall\_Midge/Gall\_Midge\_001.jpg}
\item[\textbf{Think:}] Let me also check Sooty Mould reference images:
\item[\textbf{Tool:}] \texttt{Read} $\rightarrow$ \texttt{Plant\_Disease\_Dataset/Foliar\_Disease\_Stress/Sooty\_Mould/Sooty\_Mould\_001.jpg}
\item[\textbf{Tool:}] \texttt{Read} $\rightarrow$ \texttt{Plant\_Disease\_Dataset/Foliar\_Disease\_Stress/Sooty\_Mould/Sooty\_Mould\_002.jpg}
\item[\textbf{Think:}] \#\# Analysis

Based on my examination of the target image (test\_007.jpg) and comparison with the reference images, here's my analysis:

**Target Image Observations:**
- Tan/brown colored leaf (indicating stress or maturity)
- Scattered reddish-brown spots/patches across the leaf surface
- Spots appear to have darker edges with lighter centers
- Multiple lesions distributed across the leaf
- Green midrib still visible
- Spots appear to be embedded in the leaf tissue (sunken, not raised)

**Comparison with References:**

1. **Anthracnose Reference (Anthracnose\_001)** - Shows very similar characteristics: tan/brown leaf with dark-edged lesions embedded in tissue, lighter centers, and scattered distribution. This is a strong match.

2. **Gall Midge Reference (Gall\_Midge\_001)** - Shows raised, 3D bumps on a green leaf. This does NOT match the target, which has flat lesions on a tan leaf.

3. **Sooty Mould Reference (Sooty\_Mould\_002)** - Shows many tiny dark speckles on a mostly green leaf. While there are many spots, they appear more superficial and dust-like, and the leaf background is greener than in the target image.

**Key Diagnostic Features of Target Image:**
- Lesions are EMBEDDED/SUNKEN (not raised like gall midge)
- Leaf is tan/brown (typical for advanced anthracnose)
- Dark-edged spots with lighter centers (typical anthracnose pattern)
- Scattered distribution across the leaf
- Multiple lesions indicating disease progression

The target image most closely matches **Anthracnose** based on the sunken lesions with darker edges, lighter centers, scattered distribution, and the tan-colored leaf with visible green midrib.

```json
\{"prediction": "Anthracnose"\}
```
\end{description}
\medskip\hrule\medskip
\paragraph{test\_009.jpg — \textcolor{correct}{Correct}}
Ground Truth: \textbf{Sooty\_Mould} $\rightarrow$ Prediction: \textbf{Sooty\_Mould} \\
Turns: 8 \quad Duration: 12.6s \quad Cost: \$0.0228
\begin{description}[style=unboxed,leftmargin=0.5cm]
\item[\textbf{Think:}] I'll help you classify the disease in the mango leaf image. Let me start by reading the knowledge base and examining the target image.
\item[\textbf{Tool:}] \texttt{Read} $\rightarrow$ \texttt{disease\_symptoms.md}
\item[\textbf{Tool:}] \texttt{Read} $\rightarrow$ \texttt{test\_data/Foliar\_Disease\_Stress/images/test\_009.jpg}
\item[\textbf{Think:}] Now let me examine the reference images for the diseases that best match the features I see in test\_009.jpg. The image shows a mango leaf with dark spots/lesions scattered on a green background. This could be Anthracnose, Sooty Mould, or Bacterial Canker based on initial observation. Let me view the reference images for comparison.
\item[\textbf{Tool:}] \texttt{Read} $\rightarrow$ \texttt{Plant\_Disease\_Dataset/Foliar\_Disease\_Stress/Anthracnose/Anthracnose\_001.jpg}
\item[\textbf{Tool:}] \texttt{Read} $\rightarrow$ \texttt{Plant\_Disease\_Dataset/Foliar\_Disease\_Stress/Sooty\_Mould/Sooty\_Mould\_001.jpg}
\item[\textbf{Tool:}] \texttt{Read} $\rightarrow$ \texttt{Plant\_Disease\_Dataset/Foliar\_Disease\_Stress/Bacterial\_Canker/Bacterial\_Canker\_001.jpg}
\item[\textbf{Think:}] Let me also check additional reference images to ensure accurate classification:
\item[\textbf{Tool:}] \texttt{Read} $\rightarrow$ \texttt{Plant\_Disease\_Dataset/Foliar\_Disease\_Stress/Anthracnose/Anthracnose\_002.jpg}
\item[\textbf{Tool:}] \texttt{Read} $\rightarrow$ \texttt{Plant\_Disease\_Dataset/Foliar\_Disease\_Stress/Sooty\_Mould/Sooty\_Mould\_002.jpg}
\item[\textbf{Think:}] Based on my analysis of the target image (test\_009.jpg) and comparison with the reference images, I can now provide my classification:

**Analysis:**

**Test Image (test\_009.jpg):** Shows a mango leaf with numerous small dark spots/speckles scattered across the surface on a predominantly green background. The spots appear as tiny dark specks distributed densely over the leaf.

**Comparison with references:**
- **Sooty Mould (reference images):** Matches closely - shows the characteristic superficial dark speckling/dusty coating sitting on top of green leaf tissue, with many tiny dark speckles densely scattered
- **Anthracnose (references):** Shows larger, more distinct brown/tan lesions with darker edges and sometimes holes - not matching the fine speckled pattern in test\_009
- **Bacterial Canker (reference):** Shows angular lesions with yellow halos and distinct raised spots - not present in test\_009

The target image displays the distinctive "dirty" or "sooty" appearance with fine, densely scattered dark speckles on a green leaf surface, which is the hallmark of Sooty Mould according to the knowledge base description.

```json
\{"prediction": "Sooty\_Mould"\}
```
\end{description}
\medskip\hrule\medskip
\paragraph{test\_010.jpg — \textcolor{correct}{Correct}}
Ground Truth: \textbf{Sooty\_Mould} $\rightarrow$ Prediction: \textbf{Sooty\_Mould} \\
Turns: 9 \quad Duration: 18.4s \quad Cost: \$0.0240
\begin{description}[style=unboxed,leftmargin=0.5cm]
\item[\textbf{Think:}] I'll help you classify the disease in the image. Let me start by reading the knowledge base and then examining the target image and reference images.
\item[\textbf{Tool:}] \texttt{Read} $\rightarrow$ \texttt{disease\_symptoms.md}
\item[\textbf{Tool:}] \texttt{Read} $\rightarrow$ \texttt{test\_data/Foliar\_Disease\_Stress/images/test\_010.jpg}
\item[\textbf{Think:}] Now let me view reference images for the top candidate classes. Based on the target image showing a green leaf with numerous small dark speckles/spots scattered across the surface, this appears to be either Sooty Mould or Gall Midge. Let me examine reference images:
\item[\textbf{Tool:}] \texttt{Read} $\rightarrow$ \texttt{Plant\_Disease\_Dataset/Foliar\_Disease\_Stress/Sooty\_Mould/Sooty\_Mould\_001.jpg}
\item[\textbf{Tool:}] \texttt{Read} $\rightarrow$ \texttt{Plant\_Disease\_Dataset/Foliar\_Disease\_Stress/Sooty\_Mould/Sooty\_Mould\_002.jpg}
\item[\textbf{Tool:}] \texttt{Read} $\rightarrow$ \texttt{Plant\_Disease\_Dataset/Foliar\_Disease\_Stress/Gall\_Midge/Gall\_Midge\_001.jpg}
\item[\textbf{Tool:}] \texttt{Read} $\rightarrow$ \texttt{Plant\_Disease\_Dataset/Foliar\_Disease\_Stress/Gall\_Midge/Gall\_Midge\_002.jpg}
\item[\textbf{Think:}] Let me also check Anthracnose reference images as a third candidate:
\item[\textbf{Tool:}] \texttt{Read} $\rightarrow$ \texttt{Plant\_Disease\_Dataset/Foliar\_Disease\_Stress/Anthracnose/Anthracnose\_001.jpg}
\item[\textbf{Tool:}] \texttt{Read} $\rightarrow$ \texttt{Plant\_Disease\_Dataset/Foliar\_Disease\_Stress/Anthracnose/Anthracnose\_002.jpg}
\item[\textbf{Think:}] \#\# Analysis

Based on my comparison of the target image with reference images and symptom descriptions:

**Target Image Features:**
- Green leaf background with many small, scattered dark speckles
- Speckles appear flat and superficial, sitting on the leaf surface
- Dense distribution of tiny dark spots giving a "dirty" or "dusty" appearance
- No raised bumps or 3D galls
- No distinct lesions with halos
- Overall leaf remains healthy green beneath the dark spots

**Comparison with Candidates:**

1. **Sooty Mould** (Best Match): The target image closely matches Sooty\_Mould\_002.jpg reference - it shows the characteristic superficial dark coating of many tiny speckles scattered densely across a green leaf, creating a dirty/sooty appearance. This perfectly aligns with the knowledge base description of "numerous tiny dark speckles densely scattered across the leaf giving a dirty or dusty look."

2. **Gall Midge** (Ruled Out): Reference images show distinct raised bumps/galls that are 3D and clearly protrude from the leaf surface. The target image lacks this 3D bump appearance - the spots are flat and superficial.

3. **Anthracnose** (Ruled Out): Reference images show dark brown lesions with distinct borders, some with holes and damaged leaf margins. The target image shows evenly scattered speckles without the characteristic lesion structure or tissue damage.

The dark speckles in test\_010.jpg are superficial (sitting on top of the green leaf) rather than embedded or raised, which definitively points to **Sooty Mould**.

```json
\{"prediction": "Sooty\_Mould"\}
```
\end{description}
\medskip\hrule\medskip
\paragraph{test\_011.jpg — \textcolor{correct}{Correct}}
Ground Truth: \textbf{Anthracnose} $\rightarrow$ Prediction: \textbf{Anthracnose} \\
Turns: 8 \quad Duration: 14.1s \quad Cost: \$0.0238
\begin{description}[style=unboxed,leftmargin=0.5cm]
\item[\textbf{Think:}] I'll help you classify the disease in the mango leaf image. Let me start by reading the knowledge base and examining the target image.
\item[\textbf{Tool:}] \texttt{Read} $\rightarrow$ \texttt{disease\_symptoms.md}
\item[\textbf{Tool:}] \texttt{Read} $\rightarrow$ \texttt{test\_data/Foliar\_Disease\_Stress/images/test\_011.jpg}
\item[\textbf{Think:}] Now let me examine the target image more closely. I can see a mango leaf that is primarily brown/tan colored with some areas showing red or darker coloration. The leaf appears to have discoloration affecting large portions of it. Let me view reference images for the top candidate diseases to compare.

Based on the visual features I'm seeing, the primary candidates are:
1. **Die Back** - the leaf shows browning and discoloration
2. **Anthracnose** - the brown/tan coloring with potential lesions
3. **Powdery Mildew** - though less likely due to the color pattern

Let me view reference images for these candidates:
\item[\textbf{Tool:}] \texttt{Read} $\rightarrow$ \texttt{Plant\_Disease\_Dataset/Foliar\_Disease\_Stress/Die\_Back/Die\_Back\_001.jpg}
\item[\textbf{Tool:}] \texttt{Read} $\rightarrow$ \texttt{Plant\_Disease\_Dataset/Foliar\_Disease\_Stress/Anthracnose/Anthracnose\_001.jpg}
\item[\textbf{Tool:}] \texttt{Read} $\rightarrow$ \texttt{Plant\_Disease\_Dataset/Foliar\_Disease\_Stress/Anthracnose/Anthracnose\_002.jpg}
\item[\textbf{Think:}] Good! Now let me look at one more anthracnose example and a powdery mildew reference to be thorough:
\item[\textbf{Tool:}] \texttt{Read} $\rightarrow$ \texttt{Plant\_Disease\_Dataset/Foliar\_Disease\_Stress/Anthracnose/Anthracnose\_003.jpg}
\item[\textbf{Tool:}] \texttt{Read} $\rightarrow$ \texttt{Plant\_Disease\_Dataset/Foliar\_Disease\_Stress/Powdery\_Mildew/Powdery\_Mildew\_001.jpg}
\item[\textbf{Think:}] \#\# Analysis of Test Image 011

After carefully examining the target image and comparing it with reference images for the Mango Leaf Diseases dataset, here's my analysis:

**Target Image Characteristics:**
- Leaf is predominantly brown/tan colored
- Shows significant discoloration covering most of the leaf surface
- Green midrib still visible
- The browning appears to be progressive across the leaf
- Leaf margins appear intact (not eaten away)
- No visible raised bumps (galls)
- No powdery white coating
- No distinct angular spots with halos

**Comparison with References:**

1. **Die Back** (Die\_Back\_001.jpg): Shows severely wilted, completely brown, dried leaf with curling - the test image is not as severely dried as this reference
2. **Anthracnose** (Anthracnose\_001.jpg, 002.jpg, 003.jpg): Shows:
   - Brown/tan coloration with scattered darker spots ✓
   - Green midrib visible ✓
   - Flat lesions embedded in tissue ✓
   - Some areas with lighter centers ✓
   - Matches the pattern seen in test\_011
3. **Powdery Mildew** (Powdery\_Mildew\_001.jpg): Shows white/gray coating on green leaf - the test image lacks the white powdery appearance

The test image most closely matches the **Anthracnose** disease pattern, showing the characteristic tan/brown discoloration with embedded flat lesions and the visible green midrib that distinguishes it from the more severely dried Die Back condition.

```json
\{"prediction": "Anthracnose"\}
```
\end{description}
\medskip\hrule\medskip
\paragraph{test\_012.jpg — \textcolor{correct}{Correct}}
Ground Truth: \textbf{Gall\_Midge} $\rightarrow$ Prediction: \textbf{Gall\_Midge} \\
Turns: 7 \quad Duration: 11.0s \quad Cost: \$0.0196
\begin{description}[style=unboxed,leftmargin=0.5cm]
\item[\textbf{Think:}] I'll help you classify the plant disease in the image. Let me start by reading the knowledge base and viewing the target image.
\item[\textbf{Tool:}] \texttt{Read} $\rightarrow$ \texttt{disease\_symptoms.md}
\item[\textbf{Tool:}] \texttt{Read} $\rightarrow$ \texttt{test\_data/Foliar\_Disease\_Stress/images/test\_012.jpg}
\item[\textbf{Think:}] Now let me view reference images for the top candidate diseases. Based on the target image showing what appears to be small bumps/galls on a green leaf, I should compare with Gall Midge and also check Sooty Mould and Powdery Mildew as alternatives.
\item[\textbf{Tool:}] \texttt{Read} $\rightarrow$ \texttt{Plant\_Disease\_Dataset/Foliar\_Disease\_Stress/Gall\_Midge/Gall\_Midge\_001.jpg}
\item[\textbf{Tool:}] \texttt{Read} $\rightarrow$ \texttt{Plant\_Disease\_Dataset/Foliar\_Disease\_Stress/Gall\_Midge/Gall\_Midge\_002.jpg}
\item[\textbf{Tool:}] \texttt{Read} $\rightarrow$ \texttt{Plant\_Disease\_Dataset/Foliar\_Disease\_Stress/Sooty\_Mould/Sooty\_Mould\_001.jpg}
\item[\textbf{Tool:}] \texttt{Read} $\rightarrow$ \texttt{Plant\_Disease\_Dataset/Foliar\_Disease\_Stress/Powdery\_Mildew/Powdery\_Mildew\_001.jpg}
\item[\textbf{Think:}] Perfect! I can now analyze the test image by comparing it with the reference images.

\#\# Analysis:

**Target Image (test\_012.jpg):**
- Shows a green mango leaf
- Has visible small dark spots/bumps scattered along the leaf
- The leaf background remains predominantly green
- The spots appear to have some raised/3D appearance

**Comparison with Reference Images:**

1. **Gall Midge References:** Shows raised, dome-like bumps on otherwise green leaves with clear depth and shadow. The galls are scattered sparsely on green leaf tissue. This matches very well with the target image's appearance.

2. **Sooty Mould Reference:** Shows dark coating/speckles that appear more like a superficial dust or film layer. While there are dark spots, they appear flatter and more like a coating rather than raised bumps.

3. **Powdery Mildew Reference:** Shows a distinctive white/grayish powdery coating visible in the image, which is NOT present in test\_012.jpg.

**Key Visual Features in test\_012.jpg:**
- Small raised bumps/galls visible on the leaf surface
- Scattered distribution on an otherwise green leaf
- Dark brown coloring of the galls
- Clear 3D appearance with depth (similar to Gall Midge references)

The target image matches the Gall Midge symptom description: "RAISED, 3D BUMP-LIKE GALLS that protrude visibly from the leaf surface" on a "GREEN and HEALTHY-looking leaf."

```json
\{"prediction": "Gall\_Midge"\}
```
\end{description}
\medskip\hrule\medskip

\subsubsection{Incorrect Predictions --- Error Analysis}

\paragraph{test\_003.jpg — \textcolor{wrong}{Incorrect}}
Ground Truth: \textbf{Powdery\_Mildew} $\rightarrow$ Prediction: \textbf{\textcolor{wrong}{Sooty\_Mould}} \\
Turns: 7 \quad Duration: 11.4s \quad Cost: \$0.0198
\begin{description}[style=unboxed,leftmargin=0.5cm]
\item[\textbf{Think:}] I'll help you classify the mango leaf disease in the image. Let me start by reading the knowledge base and examining the target image.
\item[\textbf{Tool:}] \texttt{Read} $\rightarrow$ \texttt{disease\_symptoms.md}
\item[\textbf{Tool:}] \texttt{Read} $\rightarrow$ \texttt{test\_data/Foliar\_Disease\_Stress/images/test\_003.jpg}
\item[\textbf{Think:}] Now let me examine the target image more carefully and view reference images for the top candidate diseases. Based on the image, I can see a mango leaf with numerous small dark speckles distributed across a greenish leaf surface. This pattern is most consistent with either Sooty Mould or Gall Midge. Let me view reference images for these candidates.
\item[\textbf{Tool:}] \texttt{Read} $\rightarrow$ \texttt{Plant\_Disease\_Dataset/Foliar\_Disease\_Stress/Sooty\_Mould/Sooty\_Mould\_001.jpg}
\item[\textbf{Tool:}] \texttt{Read} $\rightarrow$ \texttt{Plant\_Disease\_Dataset/Foliar\_Disease\_Stress/Sooty\_Mould/Sooty\_Mould\_002.jpg}
\item[\textbf{Tool:}] \texttt{Read} $\rightarrow$ \texttt{Plant\_Disease\_Dataset/Foliar\_Disease\_Stress/Gall\_Midge/Gall\_Midge\_001.jpg}
\item[\textbf{Tool:}] \texttt{Read} $\rightarrow$ \texttt{Plant\_Disease\_Dataset/Foliar\_Disease\_Stress/Gall\_Midge/Gall\_Midge\_002.jpg}
\item[\textbf{Think:}] Perfect! Now I can clearly compare the images:

**Target Image (test\_003.jpg):** Shows a mango leaf with numerous small dark speckles densely scattered across the entire greenish-yellow surface, giving it a "dirty" or "dusty" appearance.

**Comparison with references:**

1. **Sooty Mould references** (Sooty\_Mould\_001.jpg and 002.jpg):
   - Sooty\_Mould\_001: Shows dark patches/coating along the leaf margin
   - Sooty\_Mould\_002: Shows dense, numerous small dark speckles distributed across a green leaf - **MATCHES the target image very closely**
   - The key feature is the superficial dark speckles that sit on top of the green leaf, giving a sooty/dusty appearance

2. **Gall Midge references** (Gall\_Midge\_001.jpg and 002.jpg):
   - Shows raised 3D bumps/galls (visible as protrusions with depth)
   - The galls appear as distinct raised lesions, not the scattered speckles visible in the target image

**Analysis:**
The target image exhibits:
- Numerous small dark speckles/dots scattered across the leaf
- Superficial appearance (sitting on top of the leaf, not embedded)
- Dense coverage that looks like a dusty coating
- Greenish background leaf still visible
- Flat appearance (no raised bumps like Gall Midge)

This matches the Sooty Mould description perfectly: "numerous tiny dark speckles densely scattered across the leaf giving a 'dirty' or 'dusty' look" and matches Sooty\_Mould\_002.jpg reference image.

```json
\{"prediction": "Sooty\_Mould"\}
```
\end{description}
\textbf{Final Reasoning:}
\begin{quote}
\small\itshape Perfect! Now I can clearly compare the images:

**Target Image (test\_003.jpg):** Shows a mango leaf with numerous small dark speckles densely scattered across the entire greenish-yellow surface, giving it a "dirty" or "dusty" appearance.

**Comparison with references:**

1. **Sooty Mould references** (Sooty\_Mould\_001.jpg and 002.jpg):
   - Sooty\_Mould\_001: Shows dark patches/coating along the leaf margin
   - Sooty\_Mould\_002: Shows dense, numerous small dark speckles distributed across a green leaf - **MATCHES the target image very closely**
   - The key feature is the superficial dark speckles that sit on top of the green leaf, giving a sooty/dusty appearance

2. **Gall Midge references** (Gall\_Midge\_001.jpg and 002.jpg):
   - Shows raised 3D bumps/galls (visible as protrusions with depth)
   - The galls appear as distinct raised lesions, not the scattered speckles visible in the target image

**Analysis:**
The target image exhibits:
- Numerous small dark speckles/dots scattered across the leaf
- Superficial appearance (sitting on top of the leaf, not embedded)
- Dense coverage that looks like a dusty coating
- Greenish background leaf still visible
- Flat appearance (no raised bumps like Gall Midge)

This matches the Sooty Mould description perfectly: "numerous tiny dark speckles densely scattered across the leaf giving a 'dirty' or 'dusty' look" and matches Sooty\_Mould\_002.jpg reference image.

```json
\{"prediction": "Sooty\_Mould"\}
```
\end{quote}
\medskip\hrule\medskip
\paragraph{test\_006.jpg — \textcolor{wrong}{Incorrect}}
Ground Truth: \textbf{Powdery\_Mildew} $\rightarrow$ Prediction: \textbf{\textcolor{wrong}{Sooty\_Mould}} \\
Turns: 6 \quad Duration: 8.8s \quad Cost: \$0.0185
\begin{description}[style=unboxed,leftmargin=0.5cm]
\item[\textbf{Think:}] I'll help you classify the disease in the mango leaf image. Let me start by reading the knowledge base and examining the target image.
\item[\textbf{Tool:}] \texttt{Read} $\rightarrow$ \texttt{disease\_symptoms.md}
\item[\textbf{Tool:}] \texttt{Read} $\rightarrow$ \texttt{test\_data/Foliar\_Disease\_Stress/images/test\_006.jpg}
\item[\textbf{Think:}] Now let me examine reference images for the top candidate classes. Based on the target image, I can see a leaf with dark speckles/coating on a greenish surface. This looks most consistent with Sooty Mould, but let me verify by viewing reference images for my top candidates.
\item[\textbf{Tool:}] \texttt{Read} $\rightarrow$ \texttt{Plant\_Disease\_Dataset/Foliar\_Disease\_Stress/Sooty\_Mould/Sooty\_Mould\_001.jpg}
\item[\textbf{Tool:}] \texttt{Read} $\rightarrow$ \texttt{Plant\_Disease\_Dataset/Foliar\_Disease\_Stress/Sooty\_Mould/Sooty\_Mould\_002.jpg}
\item[\textbf{Tool:}] \texttt{Read} $\rightarrow$ \texttt{Plant\_Disease\_Dataset/Foliar\_Disease\_Stress/Gall\_Midge/Gall\_Midge\_001.jpg}
\item[\textbf{Think:}] Perfect! I can now compare the visual features:

**Analysis of test\_006.jpg:**
- The target image shows a leaf with dark speckles/dots densely scattered across the surface
- The overall leaf remains greenish underneath the dark material
- The dark marks appear superficial, sitting ON TOP of the leaf like a coating
- This is NOT raised bumps (ruling out Gall Midge)
- This is NOT angular lesions with halos (ruling out Bacterial Canker)
- This is NOT sunken lesions (ruling out Anthracnose)

**Comparison with reference images:**
- **Sooty\_Mould\_002.jpg**: Shows exactly this pattern - a yellowish-green leaf covered with numerous small dark dots/speckles that create a "dirty" or "dusty" appearance. This matches the target image perfectly.
- **Sooty\_Mould\_001.jpg**: Shows darker areas but is more advanced infection
- **Gall\_Midge\_001.jpg**: Shows distinct raised bumps and shot holes, which are NOT present in the target image

The target image matches the Sooty Mould description perfectly - it displays the characteristic superficial dark coating of scattered speckles on a green leaf that looks "dusty" or "unwashed."

```json
\{"prediction": "Sooty\_Mould"\}
```
\end{description}
\textbf{Final Reasoning:}
\begin{quote}
\small\itshape Perfect! I can now compare the visual features:

**Analysis of test\_006.jpg:**
- The target image shows a leaf with dark speckles/dots densely scattered across the surface
- The overall leaf remains greenish underneath the dark material
- The dark marks appear superficial, sitting ON TOP of the leaf like a coating
- This is NOT raised bumps (ruling out Gall Midge)
- This is NOT angular lesions with halos (ruling out Bacterial Canker)
- This is NOT sunken lesions (ruling out Anthracnose)

**Comparison with reference images:**
- **Sooty\_Mould\_002.jpg**: Shows exactly this pattern - a yellowish-green leaf covered with numerous small dark dots/speckles that create a "dirty" or "dusty" appearance. This matches the target image perfectly.
- **Sooty\_Mould\_001.jpg**: Shows darker areas but is more advanced infection
- **Gall\_Midge\_001.jpg**: Shows distinct raised bumps and shot holes, which are NOT present in the target image

The target image matches the Sooty Mould description perfectly - it displays the characteristic superficial dark coating of scattered speckles on a green leaf that looks "dusty" or "unwashed."

```json
\{"prediction": "Sooty\_Mould"\}
```
\end{quote}
\medskip\hrule\medskip
\paragraph{test\_008.jpg — \textcolor{wrong}{Incorrect}}
Ground Truth: \textbf{Die\_Back} $\rightarrow$ Prediction: \textbf{\textcolor{wrong}{Cutting\_Weevil}} \\
Turns: 6 \quad Duration: 10.2s \quad Cost: \$0.0184
\begin{description}[style=unboxed,leftmargin=0.5cm]
\item[\textbf{Think:}] I'll help you classify the disease in the mango leaf image. Let me start by reading the knowledge base and then analyzing the image.
\item[\textbf{Tool:}] \texttt{Read} $\rightarrow$ \texttt{disease\_symptoms.md}
\item[\textbf{Tool:}] \texttt{Read} $\rightarrow$ \texttt{test\_data/Foliar\_Disease\_Stress/images/test\_008.jpg}
\item[\textbf{Think:}] Now let me view the reference images for my top candidate diseases. Based on the image showing a severely wilted, curled, and darkened leaf, I should examine Cutting Weevil, Die Back, and Powdery Mildew as candidates.
\item[\textbf{Tool:}] \texttt{Read} $\rightarrow$ \texttt{Plant\_Disease\_Dataset/Foliar\_Disease\_Stress/Cutting\_Weevil/Cutting\_Weevil\_001.jpg}
\item[\textbf{Tool:}] \texttt{Read} $\rightarrow$ \texttt{Plant\_Disease\_Dataset/Foliar\_Disease\_Stress/Die\_Back/Die\_Back\_001.jpg}
\item[\textbf{Tool:}] \texttt{Read} $\rightarrow$ \texttt{Plant\_Disease\_Dataset/Foliar\_Disease\_Stress/Powdery\_Mildew/Powdery\_Mildew\_001.jpg}
\item[\textbf{Think:}] Excellent! The comparison is clear now. Let me analyze:

**Test Image (test\_008.jpg):**
- Single darkened, severely curled and wilted leaf
- Progressive browning and drying
- Leaf is completely drooped/curled upward
- No visible spots, lesions, or disease coating
- Shows mechanical damage/stress rather than fungal disease

**Candidate Comparison:**

1. **Cutting Weevil** (Reference image): Shows multiple leaves with brown curling, twisted shapes, and mechanical damage patterns - MATCHES closely with test image's appearance

2. **Die Back** (Reference image): Shows a single leaf with similar curling and browning, but typically presents with twig damage and defoliation patterns on branches

3. **Powdery Mildew** (Reference image): Shows white/grayish powdery coating on a relatively healthy green leaf - DOES NOT match (test image has no coating, is completely darkened)

The test image shows characteristics most consistent with **Cutting Weevil**:
- Brown, curled, and crumpled appearance
- Severe wilting and drooping
- Leaf looks mechanically damaged rather than diseased
- The distinctive curved/twisted shape matches the weevil damage pattern

```json
\{"prediction": "Cutting\_Weevil"\}
```
\end{description}
\textbf{Final Reasoning:}
\begin{quote}
\small\itshape Excellent! The comparison is clear now. Let me analyze:

**Test Image (test\_008.jpg):**
- Single darkened, severely curled and wilted leaf
- Progressive browning and drying
- Leaf is completely drooped/curled upward
- No visible spots, lesions, or disease coating
- Shows mechanical damage/stress rather than fungal disease

**Candidate Comparison:**

1. **Cutting Weevil** (Reference image): Shows multiple leaves with brown curling, twisted shapes, and mechanical damage patterns - MATCHES closely with test image's appearance

2. **Die Back** (Reference image): Shows a single leaf with similar curling and browning, but typically presents with twig damage and defoliation patterns on branches

3. **Powdery Mildew** (Reference image): Shows white/grayish powdery coating on a relatively healthy green leaf - DOES NOT match (test image has no coating, is completely darkened)

The test image shows characteristics most consistent with **Cutting Weevil**:
- Brown, curled, and crumpled appearance
- Severe wilting and drooping
- Leaf looks mechanically damaged rather than diseased
- The distinctive curved/twisted shape matches the weevil damage pattern

```json
\{"prediction": "Cutting\_Weevil"\}
```
\end{quote}
\medskip\hrule\medskip
\paragraph{test\_013.jpg — \textcolor{wrong}{Incorrect}}
Ground Truth: \textbf{Gall\_Midge} $\rightarrow$ Prediction: \textbf{\textcolor{wrong}{Sooty\_Mould}} \\
Turns: 8 \quad Duration: 14.6s \quad Cost: \$0.0211
\begin{description}[style=unboxed,leftmargin=0.5cm]
\item[\textbf{Think:}] I'll help you classify the mango leaf disease. Let me start by examining the knowledge base and the target image.
\item[\textbf{Tool:}] \texttt{Read} $\rightarrow$ \texttt{disease\_symptoms.md}
\item[\textbf{Tool:}] \texttt{Read} $\rightarrow$ \texttt{test\_data/Foliar\_Disease\_Stress/images/test\_013.jpg}
\item[\textbf{Think:}] Good! Now let me examine the target image more carefully. I can see a green mango leaf with several small dark brown spots scattered across it. The leaf background is predominantly green. Let me view the reference images for the most likely candidates based on the visual features I'm observing.

The target image shows:
- Green leaf background
- Small dark brown spots scattered across the leaf surface
- Spots appear relatively small and distributed

The main candidates appear to be:
1. **Gall Midge** - raised bumps on green leaf
2. **Sooty Mould** - superficial dark coating/spots on green leaf
3. **Anthracnose** - flat sunken lesions

Let me view the reference images for these candidates:
\item[\textbf{Tool:}] \texttt{Read} $\rightarrow$ \texttt{Plant\_Disease\_Dataset/Foliar\_Disease\_Stress/Gall\_Midge/Gall\_Midge\_001.jpg}
\item[\textbf{Tool:}] \texttt{Read} $\rightarrow$ \texttt{Plant\_Disease\_Dataset/Foliar\_Disease\_Stress/Gall\_Midge/Gall\_Midge\_002.jpg}
\item[\textbf{Tool:}] \texttt{Read} $\rightarrow$ \texttt{Plant\_Disease\_Dataset/Foliar\_Disease\_Stress/Sooty\_Mould/Sooty\_Mould\_001.jpg}
\item[\textbf{Tool:}] \texttt{Read} $\rightarrow$ \texttt{Plant\_Disease\_Dataset/Foliar\_Disease\_Stress/Sooty\_Mould/Sooty\_Mould\_002.jpg}
\item[\textbf{Tool:}] \texttt{Read} $\rightarrow$ \texttt{Plant\_Disease\_Dataset/Foliar\_Disease\_Stress/Anthracnose/Anthracnose\_001.jpg}
\item[\textbf{Think:}] Perfect! Now I can make a detailed comparison:

**Target Image Analysis (test\_013.jpg):**
- Green leaf background with vibrant color
- Multiple small dark brown spots scattered across the leaf surface
- Spots appear superficial and distributed as small specks
- The spots look like they sit ON TOP of the leaf surface
- Overall leaf structure is intact

**Reference Image Comparison:**

1. **Gall\_Midge\_001.jpg**: Shows raised, 3D dome-shaped galls that are clearly visible bumps/protrusions. Very distinct from target.

2. **Gall\_Midge\_002.jpg**: Similar raised gall appearance with visible depth. Not matching target image.

3. **Sooty\_Mould\_001.jpg**: Shows dark coating/patches on a green leaf with a dirty, sooty appearance - very similar to target.

4. **Sooty\_Mould\_002.jpg**: Shows extensive scattered dark specks densely distributed across a green leaf - this matches the target image very well! The spots appear as tiny dark speckles sitting on the leaf surface.

5. **Anthracnose\_001.jpg**: Shows large brown patches and discoloration covering significant portions of the leaf - very different from target.

**Conclusion:**
The target image (test\_013.jpg) best matches **Sooty\_Mould**. The key features that match are:
- Green leaf background remains vibrant
- Numerous small dark brown specks/spots scattered across the surface
- Spots appear superficial, sitting ON TOP of the leaf (not embedded)
- No raised bumps (unlike Gall Midge)
- No large lesions or significant browning (unlike Anthracnose)
- The overall appearance is of a "dirty" or "sooty" coating

```json
\{"prediction": "Sooty\_Mould"\}
```
\end{description}
\textbf{Final Reasoning:}
\begin{quote}
\small\itshape Perfect! Now I can make a detailed comparison:

**Target Image Analysis (test\_013.jpg):**
- Green leaf background with vibrant color
- Multiple small dark brown spots scattered across the leaf surface
- Spots appear superficial and distributed as small specks
- The spots look like they sit ON TOP of the leaf surface
- Overall leaf structure is intact

**Reference Image Comparison:**

1. **Gall\_Midge\_001.jpg**: Shows raised, 3D dome-shaped galls that are clearly visible bumps/protrusions. Very distinct from target.

2. **Gall\_Midge\_002.jpg**: Similar raised gall appearance with visible depth. Not matching target image.

3. **Sooty\_Mould\_001.jpg**: Shows dark coating/patches on a green leaf with a dirty, sooty appearance - very similar to target.

4. **Sooty\_Mould\_002.jpg**: Shows extensive scattered dark specks densely distributed across a green leaf - this matches the target image very well! The spots appear as tiny dark speckles sitting on the leaf surface.

5. **Anthracnose\_001.jpg**: Shows large brown patches and discoloration covering significant portions of the leaf - very different from target.

**Conclusion:**
The target image (test\_013.jpg) best matches **Sooty\_Mould**. The key features that match are:
- Green leaf background remains vibrant
- Numerous small dark brown specks/spots scattered across the surface
- Spots appear superficial, sitting ON TOP of the leaf (not embedded)
- No raised bumps (unlike Gall Midge)
- No large lesions or significant browning (unlike Anthracnose)
- The overall appearance is of a "dirty" or "sooty" coating

```json
\{"prediction": "Sooty\_Mould"\}
```
\end{quote}
\medskip\hrule\medskip
\paragraph{test\_014.jpg — \textcolor{wrong}{Incorrect}}
Ground Truth: \textbf{Bacterial\_Canker} $\rightarrow$ Prediction: \textbf{\textcolor{wrong}{Sooty\_Mould}} \\
Turns: 8 \quad Duration: 12.1s \quad Cost: \$0.0232
\begin{description}[style=unboxed,leftmargin=0.5cm]
\item[\textbf{Think:}] I'll help you classify the plant disease. Let me start by reading the knowledge base and examining the target image.
\item[\textbf{Tool:}] \texttt{Read} $\rightarrow$ \texttt{disease\_symptoms.md}
\item[\textbf{Tool:}] \texttt{Read} $\rightarrow$ \texttt{test\_data/Foliar\_Disease\_Stress/images/test\_014.jpg}
\item[\textbf{Think:}] Good! I can see the target image (test\_014.jpg) shows a green mango leaf with some visible damage/spots. Now let me examine the reference images for the top candidate diseases to make an accurate classification. Based on the image, I can see some dark spots on the leaf. Let me check the reference images for the most likely candidates:
\item[\textbf{Tool:}] \texttt{Read} $\rightarrow$ \texttt{Plant\_Disease\_Dataset/Foliar\_Disease\_Stress/Anthracnose/Anthracnose\_001.jpg}
\item[\textbf{Tool:}] \texttt{Read} $\rightarrow$ \texttt{Plant\_Disease\_Dataset/Foliar\_Disease\_Stress/Bacterial\_Canker/Bacterial\_Canker\_001.jpg}
\item[\textbf{Tool:}] \texttt{Read} $\rightarrow$ \texttt{Plant\_Disease\_Dataset/Foliar\_Disease\_Stress/Sooty\_Mould/Sooty\_Mould\_001.jpg}
\item[\textbf{Think:}] Now let me check a couple more reference images to confirm:
\item[\textbf{Tool:}] \texttt{Read} $\rightarrow$ \texttt{Plant\_Disease\_Dataset/Foliar\_Disease\_Stress/Gall\_Midge/Gall\_Midge\_001.jpg}
\item[\textbf{Tool:}] \texttt{Read} $\rightarrow$ \texttt{Plant\_Disease\_Dataset/Foliar\_Disease\_Stress/Anthracnose/Anthracnose\_002.jpg}
\item[\textbf{Think:}] Based on my analysis of the target image (test\_014.jpg) and comparison with reference images:

**Visual Analysis of test\_014.jpg:**
- Green leaf with visible brown/tan spots scattered on the surface
- The spots appear to be small, dark lesions
- The leaf background remains mostly green
- The spots look like they have some texture/depth to them

**Comparison with Reference Images:**

1. **Anthracnose** (Anthracnose\_001, Anthracnose\_002): Shows dark, sunken lesions with irregular shapes and holes. The reference images show more severe browning and larger lesions than what I see in test\_014.

2. **Bacterial\_Canker** (Bacterial\_Canker\_001): Shows dark spots with yellow halos and angular shapes. The test\_014 image does not show prominent yellow halos around the spots.

3. **Sooty\_Mould** (Sooty\_Mould\_001): Shows a dark coating or smudging on the leaf surface. This matches the appearance in test\_014 - small dark speckles scattered across a green leaf, giving a "dirty" or "dusty" appearance.

4. **Gall\_Midge** (Gall\_Midge\_001): Shows raised bump-like galls that are 3D and cast shadows. The spots in test\_014 appear flat, not raised bumps.

The target image best matches **Sooty\_Mould** - it shows multiple small dark speckles distributed across a green leaf surface, creating a superficial dirty/dusty appearance characteristic of sooty mould.

```json
\{"prediction": "Sooty\_Mould"\}
```
\end{description}
\textbf{Final Reasoning:}
\begin{quote}
\small\itshape Based on my analysis of the target image (test\_014.jpg) and comparison with reference images:

**Visual Analysis of test\_014.jpg:**
- Green leaf with visible brown/tan spots scattered on the surface
- The spots appear to be small, dark lesions
- The leaf background remains mostly green
- The spots look like they have some texture/depth to them

**Comparison with Reference Images:**

1. **Anthracnose** (Anthracnose\_001, Anthracnose\_002): Shows dark, sunken lesions with irregular shapes and holes. The reference images show more severe browning and larger lesions than what I see in test\_014.

2. **Bacterial\_Canker** (Bacterial\_Canker\_001): Shows dark spots with yellow halos and angular shapes. The test\_014 image does not show prominent yellow halos around the spots.

3. **Sooty\_Mould** (Sooty\_Mould\_001): Shows a dark coating or smudging on the leaf surface. This matches the appearance in test\_014 - small dark speckles scattered across a green leaf, giving a "dirty" or "dusty" appearance.

4. **Gall\_Midge** (Gall\_Midge\_001): Shows raised bump-like galls that are 3D and cast shadows. The spots in test\_014 appear flat, not raised bumps.

The target image best matches **Sooty\_Mould** - it shows multiple small dark speckles distributed across a green leaf surface, creating a superficial dirty/dusty appearance characteristic of sooty mould.

```json
\{"prediction": "Sooty\_Mould"\}
```
\end{quote}
\medskip\hrule\medskip
\newpage
\end{document}